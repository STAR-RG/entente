\documentclass[11pt]{article}
%\documentclass[dvips,11pt]{article}

% \usepackage{natbib}
% \setlength{\bibsep}{0pt plus 0.3ex}
\usepackage{titlesec}
\titlespacing*{\section}{0pt}{0.3\baselineskip}{0.3\baselineskip}
%,nohead,nofoot
%\usepackage[margin=1n,left=1cm,top=1cm,right=1.75cm,bottom=1.5cm]{geometry}
\usepackage[margin=1in,includeheadfoot,left=1cm,top=0.5cm,right=1.5cm,bottom=0.5cm]{geometry}
%\usepackage[margin=1in,includeheadfoot]{geometry}
\usepackage{url}
\usepackage{cite}
\usepackage{lipsum}
\usepackage{booktabs}
\usepackage{tabularx}

% ADD THE FOLLOWING COUPLE LINES INTO YOUR PREAMBLE
\let\OLDthebibliography\thebibliography
\renewcommand\thebibliography[1]{
  \OLDthebibliography{#1}
  \setlength{\parskip}{0pt}
  \setlength{\itemsep}{0pt plus 0.3ex}
}

\usepackage[pdftex]{graphicx}
\usepackage{url}
\usepackage{color}
\definecolor{darkred}{rgb}{0.5,0,0}
\definecolor{darkgreen}{rgb}{0,0.5,0}
\definecolor{darkblue}{rgb}{0,0,0.5}
%% macros
\newcommand{\splat}{SPLat}
\newcommand{\ie}{i.e.}
\newcommand{\eg}{e.g.}
\newcommand{\Fix}[1]{\textbf{[[}{\color{red} #1}\textbf{]]}}
\newcommand{\Comment}[1]{}

%% \setlength{\oddsidemargin}{0.25in}
%% \setlength{\textwidth}{6.5in}
%% \setlength{\topmargin}{0in}
%% \setlength{\textheight}{8.5in}

%\renewcommand\Affilfont{\fontsize{9}{10.8}\itshape}
% These force using more of the margins that is the default style

\begin{document}

\title{Finding Bugs in JavaScript Engines with Differential Testing}
\author{}
\date{}

% You can leave out "date" and it will be added automatically for today
% You can change the "\today" date to any text you like

\makeatletter
\def\maketitle{%
  \par{\centering\large\textbf{\@title}\normalsize\par}\vspace{3ex}%
  \par{\@author}%
  \par}
\makeatother

\maketitle

\vspace{-2ex}
\small    
\begin{table}[h!] 
  \centering%% Centro de Inform\'atica - CIn \\
  \begin{tabular*}{.85\linewidth}{l}

    Marcelo d'Amorim and Igor Sim\~oes (M.Sc. student)\\
    
    Postal Address: Av. Jornalista Anibal Fernandes, S/N. Cidade
    Universit\'aria, PE, Brazil, 50.740-560 \\
    
    Email addresses: \{damorim, isol2\}@cin.ufpe.br\\

    Phones: +55 (81) 2126-8430,
    ext. 4379 [work], +55 (81) 98800-2010 [mobile]\\

    Affiliation: Federal University of Pernambuco, Department of
    Computer Science.\\

    Contacts at Google:~\Fix{...}

  \end{tabular*}
\end{table}
\normalsize

\vspace{-2ex}
\begin{abstract}
...
\end{abstract}


\section{Problem}

\Fix{...}

\vspace{-1ex}
\begin{center}
\fbox{
  \begin{minipage}{11.75cm}
    \centering
    \textit{...}
  \end{minipage}
}
\end{center}
\vspace{-1ex}

\vspace{1ex} \textbf{Related Work.}
~According Maddox \cite{maddox-2015-acmqueue}, is very complex to testing 
distributed system that have multiple components due lack of configurations, 
data-flow between nodes and asynchronous behaviour. (wip)

Fuzzing is a very common technique used in software engineering for
generate code mutations from existing test sample to improve the black-box testing method.
In Yang et al. \cite{yang-2011-finding}, it was designed the 
CSmith\footnote{\url{http://embed.cs.utah.edu/csmith/}}, a grammar-fuzzer 
based on C language to generate invalid entries and found bugs in several C compilers.
Similar fuzzers involving grammar and rules are found in Holler et al. \cite{holler-2012-fuzzing} 
that exposes the LangFuzz to generate entries based in code fragments 
(language grammar, sample code, test suite) and the tool Mozilla 
Funfuzz\footnote{\url{https://github.com/MozillaSecurity/funfuzz}}
that implements a fuzzer based on JavaScript language to improve the 
testing for SpiderMonkey engine, the interpreter of Mozilla Firefox.
\Fix{...}


\section{Design}

...

\vspace{1ex}\noindent\textbf{Input/Output.}~...

\section{Current Results}

\Fix{list/describe/explain some of the bugs we found here}

\section{Data Policy}

The results produced with this research will be made available to the
public and to the research community.  All tools and data sets
developed will be available online, and the software will be released
under an open source license.


\footnotesize
\bibliographystyle{plain}
\bibliography{tmp}

\end{document}

