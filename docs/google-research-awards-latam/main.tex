\documentclass[11pt]{article}
%\documentclass[dvips,11pt]{article}

% \usepackage{natbib}
% \setlength{\bibsep}{0pt plus 0.3ex}
\usepackage[margin=1cm]{caption}
\usepackage{titlesec}
\titlespacing*{\section}{0pt}{0.3\baselineskip}{0.3\baselineskip}
%,nohead,nofoot
%\usepackage[margin=1n,left=1cm,top=1cm,right=1.75cm,bottom=1.5cm]{geometry}
\usepackage[margin=1in,includeheadfoot,left=1cm,top=0.5cm,right=1.5cm,bottom=0.5cm]{geometry}
%\usepackage[margin=1in,includeheadfoot]{geometry}
\usepackage{url}
\usepackage{cite}
\usepackage{multirow}
\usepackage{lipsum}
\usepackage{booktabs}
\usepackage{tabularx}
\usepackage{wrapfig}
\usepackage{microtype}
\usepackage{hyperref}
\usepackage{listings}
\usepackage{subfigure}
% ADD THE FOLLOWING COUPLE LINES INTO YOUR PREAMBLE
\let\OLDthebibliography\thebibliography
\renewcommand\thebibliography[1]{
  \OLDthebibliography{#1}
  \setlength{\parskip}{0pt}
  \setlength{\itemsep}{0pt plus 0.3ex}
}

\usepackage[pdftex]{graphicx}
\usepackage{url}
\usepackage{color}
\usepackage{xcolor}
\definecolor{darkred}{rgb}{0.5,0,0}
\definecolor{darkgreen}{rgb}{0,0.5,0}
\definecolor{darkblue}{rgb}{0,0,0.5}
\definecolor{darkaquamarine}{rgb}{0.2,0.4,0.3}
%% macros
\newcommand{\ie}{i.e.}
\newcommand{\eg}{e.g.}
\newcommand{\Fix}[1]{\textbf{[[}{\color{red} #1}\textbf{]]}}
\newcommand{\Mar}[1]{\textbf{[[}{\color{blue} #1}\textbf{]]}}
\newcommand{\MAB}[1]{\textbf{[[}{\color{darkgreen} #1}\textbf{]]}}
\newcommand{\Igor}[1]{\textbf{[[}{\color{darkaquamarine} #1}\textbf{]]}}
\newcommand{\Comment}[1]{}
\newcommand{\CodeIn}[1]{{\small\texttt{#1}}}

%% \setlength{\oddsidemargin}{0.25in}
%% \setlength{\textwidth}{6.5in}
%% \setlength{\topmargin}{0in}
%% \setlength{\textheight}{8.5in}

%\renewcommand\Affilfont{\fontsize{9}{10.8}\itshape}
% These force using more of the margins that is the default style

\begin{document}

\title{Finding Bugs in JavaScript Engines with Differential Testing}
\author{}
\date{}

% You can leave out "date" and it will be added automatically for today
% You can change the "\today" date to any text you like

\makeatletter
\def\maketitle{%
  \par{\centering\large\textbf{\@title}\normalsize\par}\vspace{3ex}%
  \par{\@author}%
  \par}
\makeatother

\maketitle

\vspace{-2ex}
\small    
\begin{table}[h!] 
  \centering%% Centro de Inform\'atica - CIn \\
  \begin{tabular*}{.85\linewidth}{l}

    Marcelo d'Amorim and Igor Sim\~oes (M.Sc. student)\\
    
    Postal Address: Av. Jornalista Anibal Fernandes, S/N. Cidade
    Universit\'aria, PE, Brazil, 50.740-560 \\
    
    Email addresses: \{damorim, isol2\}@cin.ufpe.br\\

    Phones: +55 (81) 2126-8430,
    ext. 4379 [work], +55 (81) 98800-2010 [mobile]\\

    Affiliation: Federal University of Pernambuco, Department of
    Computer Science\\

    Contact at Google:~Martin Barbella (mbarbella@google.com) and Vilas Jagannath (vilasshekharbj@gmail.com)

  \end{tabular*}
\end{table}
\normalsize

\vspace{-2ex}
\begin{abstract}
JavaScript (JS) is a fundamental programming language for the
web. Finding bugs in JavaScript engines is an important
problem. Differential testing of JS engines is a promising approach to
that end. This proposal describes our testing infrastructure, current
results, and some upcoming features to improve the bug-finding
process.
\end{abstract}

\section{Problem Space}

%% Its niche is front-end and
%% back-end web development\Comment{, supported several frameworks and
%%   runtimes (\eg{}, Express.js and Node.js)}.

%% In fact, bugs in JS implementations are often reported by
%% the general public, some of which were reported by our team.
%% (confirmed by engine developers; see
%% Section~\ref{sec:results}) using a prototype of our solution (see
%% Section~\ref{sec:design}).

JavaScript (JS) is one of the most popular programming languages of
today~\cite{business-insider,stackify} largely because of its
widespread use on the web. As common in software, the JS specification
changes regularly\footnote{Recently, the specification went through a
  bigger change compared to prior releases, called by the name of
  ECMAScript6 (ES6)~\cite{es6-features}} to accommodate the pressing
demands of the community and these changes often entail sensible
changes in engine implementations~\cite{kangax} that could lead to
errors, including regressions. Automated test generation can help reduce the
cost of testing, but the lack of executable specifications is an
important obstacle. In particular, it is an obstable to find functional errors.

%\vspace{-1ex}
\begin{center}
\fbox{
  \begin{minipage}{13.5cm}
    \centering \textit{The goal of this project is to find bugs in
      implementations of JavaScript engines by leveraging diversity
      across implementations.}
  \end{minipage}
}
\end{center}
%\vspace{-1ex}

Differential testing\cite{McKeeman98differentialtesting} has been
applied in a variety of
contexts\cite{Brumley-etal-ss07,Yang-etal-pldi11,Chen-etal-fse2015,Argyros-etla-ccs16,Chen-etal-pldi16,petsios-etal-sp2017,SivakornAPKJ17}
to mitigate with the lack of oracles problem,
but it has not been thoroughly explored to find functional bugs in JS
engines. The closest work was done by Patra and
Pradel~\cite{patra2016learning}, where they evaluated their proposed
language-agnostic fuzzer to find cross-browser HTML+JS
discrepancies. This project aims to provide an adaptable
infrastructure for differential testing of runtime engines, such as
the JS engine or WebAssembly's. The sensible parts of the
infrastructure are the checks of input validity (as to reduce
waste/cost) and output correctness (as to reduce false positives).


\section{Design Space}
\label{sec:design}

\begin{wrapfigure}[10]{r}[0pt]{0.45\textwidth}
  \vspace{-5ex}
  \centering
  \includegraphics[trim=20 350 200 0,clip,width=0.45\textwidth]{google-awards-workflow}
  \caption{\label{fig:workflow}Fuzzing Infrastructure.}
  \vspace{-5ex}  
\end{wrapfigure}
Fuzzing is a popular technique to generate new test inputs from
existing inputs~\cite{fuzz-testing-history} with the typical goal of
finding crashes. Several fuzzing methods have been proposed in the
past, varying with respect to how new inputs are generated (\eg{},
coverage-based~\cite{afl,honggfuzz},
grammar-based~\cite{grammarinator,jsfunfuzz}, and
random-based~\cite{radamsa}) and whether or not coverage information
is taken into account in the fuzzing
process. Figure~\ref{fig:workflow} shows a generic pipeline of a
fuzzer infrastructure, instantiated to our problem. Arrows indicate data flow; lines
with no arrow denote encapsulation. In our case, JS files are
provided as seeded inputs to bootstrap the process. These files can
originate from the test suites of the engine codebases or from the bug
reports in their issue trackers. The fuzzing infrastructure selects
randomly one of the inputs to fuzz with a given tool, say the
black-box fuzzer Radamsa~\cite{radamsa}. Each fuzzing iteration
produces a new file. For grammar-agnostic fuzzers, like Radamsa, the
infrastructure can discard syntactically invalid
inputs using an external parser for the input language; semantically
invalid inputs can be prevented with with linters. Once the input is
found to be likely valid, the infrastructure
invokes the oracle to check correctness. In case all engines
consider the test output consistent (\ie{}, all pass or all fail), the
infrastructure discards the corresponding input. Otherwise, it considers that
input as potentially fault-revealing and interesting. Currently, we
classify alarms in two groups: HI and LO. The HI group includes
those warnings that manifest execution errors spawned directly by the
test (or its close neighborhood) whereas the LO group includes the
rest of the warnings. We found empirically that \Fix{XX\%} of the
warnings analyzed in the HI group were confirmed.

% \section{Related Work}
% Differential testing with fuzzing can improve the quality of programs
% Yang et al. \cite{yang-2011-finding}

%% \Mar{rule of thumb: explain the
%%   contribution/novelty, then results achieved, and then how it relates
%% with our proposal.}
%% \Mar{this is not clear$\rightarrow$}Yang et al. \cite{yang-2011-finding} proposed CSmith\footnote{\url{http://embed.cs.utah.edu/csmith/}}, a grammar-fuzzer 
%% of C programs to generate invalid entries\Fix{the goal of a
%%    fuzzer is to generate valid inputs/entries. this is strange}\MAB{the
%%   goal is to generate valid programs without undefined behavior to
%%   facilitate differential testing} and
%% find bugs in several C compilers\Fix{it is strange that you don't mention differential
%%   testing, which is central in CSmith, and the use of LLVM and GCC as
%%   oracles.}.\Mar{this is just listing fuzzers. don't do this. try to
%%   follow rule of thumb above.}Similar fuzzers involving grammar and rules are found in Holler et al. \cite{holler-2012-fuzzing} 
%% that exposes the LangFuzz to generate entries based in code fragments 
%% (language grammar, sample code, test suite) and the tool Mozilla 
%% Funfuzz\footnote{\url{https://github.com/MozillaSecurity/funfuzz}}
%% that implements a fuzzer based on JavaScript language to improve the 
%% testing for SpiderMonkey engine, the interpreter of Mozilla Firefox.
%% \Fix{...}

\section{Current Results}
\label{sec:results}

%% Although there are many features yet to implement in our
%% infrastructure, 

The basic workflow described in Section~\ref{sec:design} is functional. This section shows preliminary
results obtained with our infrastructure providing some evidence that finding bugs in runtime engines is
promising. Table~\ref{tab:bugs} shows the list of bugs we reported on
issue trackers of different engines in the period of a month. For
example, bug \# 6 was manifested by the JS code \CodeIn{var a =
  \{valueOf:~function()\{return ``\textbackslash{}x00''\}\} assert(+a
  === 0)\}}. 
  The \CodeIn{valueOf} is an object property that returns the primitive value
  of the specific object\cite{valueof} and in the original version of this code,
  the \CodeIn{valueOf} function returns an empty string.
  The modified code returns a string representation of an embedded null terminator
  value. 
  The unary plus ``\CodeIn{+a}" is equivalent to \CodeIn{Object.ToNumber(a.valueOf())} 
  that converts a variable to a number, otherwise the return should be the 
  value NaN (Not a Number) \cite{unary-plus}.
  As such, the engines evaluate that unary plus as a NaN value 
  at the assertion expression; hence test execution is expected to fail. 
  Chakra, however, is able to parse the string as a number zero, making the test to pass.

\newcommand{\checkm}{Y}
\newcommand{\crossmark}{N}
\begin{wraptable}[20]{tr}[0pt]{0.5\textwidth}
  \vspace{-2ex}
  \scriptsize
  \centering
  \caption{List of bug reports issued by our team from April 12 to May
    24, 2018.}
  \label{tab:bugs}
  \begin{tabular}{cccccc}
    \toprule
    Issue\#    & Date & Fuzz & Engine  & Status  & \multicolumn{1}{c}{Url}   \\
    \midrule    
    1  & 4/12 & \checkm & Chakra   & \textbf{Fixed}  & \href{https://github.com/Microsoft/ChakraCore/issues/4978}{\#4978}      \\ 
    2  & 4/12 & \checkm & Chakra   & Rejected  & \href{https://github.com/Microsoft/ChakraCore/issues/4979}{\#4979}      \\
    3  & 4/14 & \checkm & JavascriptCore  & New & \href{https://bugs.webkit.org/show\_bug.cgi?id=184629}{\#184629}        \\
    4  & 4/18 & \crossmark & JavascriptCore  & New  & \href{https://bugs.webkit.org/show\_bug.cgi?id=184749}{\#184749}        \\
    5  & 4/23 & \crossmark & Chakra  & \textbf{Confirmed}  & \href{https://github.com/Microsoft/ChakraCore/issues/5033}{\#5033}       \\
    6  & 4/25 & \checkm & Chakra  & \textbf{Fixed}     & \href{https://github.com/Microsoft/ChakraCore/issues/5038}{\#5038}      \\
    7  & 4/29 & \crossmark & Chakra  & \textbf{Confirmed}   &
    \href{https://github.com/Microsoft/ChakraCore/issues/5065}{\#5065}
    \\
    \midrule
    \multirow{2}{*}{8}  & \multirow{2}{*}{4/29} &  \multirow{2}{*}{\crossmark} & Chakra & \textbf{Confirmed} &    \href{https://github.com/Microsoft/ChakraCore/issues/5067}{\#5067} \\
                        &  &                       &
    JavascriptCore & New &    \href{https://bugs.webkit.org/show\_bug.cgi?id=185130}{\#185130}    \\
    \midrule    
    9  & 4/29 & \checkm & JavascriptCore  & New  &    \href{https://bugs.webkit.org/show\_bug.cgi?id=185127}{\#185127}    \\
    \midrule    
    \multirow{2}{*}{10} & \multirow{2}{*}{4/30}  & \multirow{2}{*}{\checkm} & Chakra & \textbf{Confirmed} &    \href{https://github.com/Microsoft/ChakraCore/issues/5076}{\#5076} \\    
                        &                        &        &
    JavascriptCore & New &
    \href{https://bugs.webkit.org/show\_bug.cgi?id=185156}{\#185156}
    \\
    \midrule    
    11 & 5/02 & \checkm & JavascriptCore  & New & \href{https://bugs.webkit.org/show\_bug.cgi?id=185197}{\#185197}\\
    12 & 5/02 & \crossmark & JavascriptCore & New  & \href{https://bugs.webkit.org/show\_bug.cgi?id=185208}{\#185208}\\
    13 & 5/10 & \checkm & Chakra & \textbf{Confirmed} & \href{https://github.com/Microsoft/ChakraCore/issues/5128}{\#5128} \\
    14 & 5/17 & \checkm & Chakra & \textbf{Confirmed} & \href{https://github.com/Microsoft/ChakraCore/issues/5182}{\#5182} \\
    15 & 5/17 & \crossmark & Chakra & \textbf{Confirmed} & \href{https://github.com/Microsoft/ChakraCore/issues/5187}{\#5187} \\
    16 & 5/21 & \crossmark & Chakra & \textbf{Confirmed} & \href{https://github.com/Microsoft/ChakraCore/issues/5203}{\#5203} \\
    17 & 5/24 & \checkm & JavascriptCore & New  & \href{https://bugs.webkit.org/show\_bug.cgi?id=185943}{\#185943}\\
   \bottomrule     
  \end{tabular}
\end{wraptable}

So far, eight of the bugs we reported
were confirmed, two of which were fixed. One bug report we
submitted was rejected on the basis that the offending JS file
manifested an expected incompatibility across engine
implementations.
Note from the table that, with the exception of one case, all bug
reports still waiting for confirmation are associated with the
JavaScriptCore engine (JSC). A closer look at the JSC issue tracker
showed that the triage process is very slow for that engine. As of
now, we did not find any bug on SpiderMonkey and V8. Finally, it is
worth noting that six of the 13 JS files that manifested
discrepancies were \emph{not} produced with fuzzing (column
``Fuzz''). These are test files from suites of different engines. This
observation emphasizes the importance of continuously collecting test suites from
multiple sources; today, we use test suites from six different open
source engines, including a total of \Fix{38.9K} test files.

%% \Igor{
%%   Nowadays, we used six javascript engines suites with approximately 2K of javascript files 
%%   that could be fuzzed. Currently, using our infrastructure we reported \Fix{13}
%%   bugs so far. The Table~\ref{tab:bugs} exposes all bugs found and reported by our team.
%%   At this moment, the status are: \Fix{6} issues was tagged as new, \Fix{4} was confirmed
%%   and \Fix{2} of them was merged and closed. Unfortunally, \Fix{1} of them was rejected
%%   due the engine does not follows the specification and dozens of others warnings already been
%%   reported previously.

%%   The Figure~\ref{fig:bug-chakra} exposes a fuzzed file generated by radamsa in our environment, it was observed
%%   that the fuzzer replace the empty string at Line 3 and replace it with the hexadecimal value "\textbackslash x00".
%%   The bug manifests in Line 7, according to ES6 specification on Mozilla dev website
%%   \footnote{Unary Plus \url{https://developer.mozilla.org/en-US/docs/Web/JavaScript/Reference/Operators/Arithmetic_Operators}},
%%   the unary plus converts the variable to an integer if it is a valid value
%%   otherwise returns NaN. The Figure~\ref{fig:pattern} shows the log output from our infrastructure,
%%   see that we set automatically this testcase as high priority because it is a failure by assertion.
%%   The differential test across all engines, shows a unexpected behaviour of ChakraCore engine that passes in this test case.
%%   It was necessary a manual inspection to check if it is a real bug, we observed that 
%%   ChakraCore indentify "\textbackslash x00" as a null terminator instead of invalid parser object making
%%   the unary plus converts null to 0.
%% }
%% \Fix{nao estou conseguindo mesclar dois listings com subfigure, por favor alguem tenta fazer isso por mim.}
%% \Igor{Coloco outro caso de bug reportado?}

%\lstdefinelanguage{JavaScript}{
  keywords={break, case, catch, continue, debugger, default, delete, do, else, finally, for, function, if, in, instanceof, new, return, switch, this, throw, try, typeof, var, void, while, with},
  morecomment=[l]{//},
  morecomment=[s]{/*}{*/},
  morestring=[b]',
  morestring=[b]",
  sensitive=true
}

\begin{figure}[htbp]
    \centering
    \begin{subfigure}[b]{0.4}
        \centering
        \scriptsize
        \lstset{
            language=JavaScript,
            keywords={function, return},
            escapeinside={@}{@},
            numbers=left,xleftmargin=1em,frame=single,framexleftmargin=0.5em,
            basicstyle=\ttfamily\scriptsize, boxpos=c,
            numberstyle=\tiny,
            linewidth=6cm
        }
\begin{lstlisting} 
var a = {
    valueOf: function () {
-    return ""
+    return "\x00"
    }
}
assert(+a === 0)
\end{lstlisting}
        \normalsize
        \caption{
            \label{fig:bug-chakra}Fuzzed file shows an unexpected behaviour on ChakraCore
        }        
    \end{subfigure}
    \begin{subfigure}[b]{0.4}
\begin{lstlisting}[basicstyle=\ttfamily]
Priority: HIGH
Pattern:
-----JavaScriptCore
Error: Test failed
-----Chakra
-----SpiderMonkey
Error: Test failed
-----v8
Error: Test failed
\end{lstlisting}
\caption{\label{fig:pattern}\Fix{JSFuzz} output log}
\end{subfigure}
\end{figure}
  

%\begin{table}[]
    \centering
    \caption{Bugs found and reported by \Fix{JSFuzz} team}
    \label{tab:bugs}
    \begin{tabular}{|c|c|c|c|}
    \hline
    Issue    & Engine                                                          & Status                                                                            & \multicolumn{1}{c|}{Url}                                                                                                                      \\ \hline
    Bug \#1  & Chakra                                                          & Fixed                                                                             & \href{https://github.com/Microsoft/ChakraCore/issues/4978}{\#4978}                                                                                           \\ \hline
    Bug \#2  & Chakra                                                          & Rejected                                                                          & \href{https://github.com/Microsoft/ChakraCore/issues/4979}{\#4979}                                                                              \\ \hline
    Bug \#3  & JavascriptCore                                                  & New                                                                               & \href{https://bugs.webkit.org/show\_bug.cgi?id=184629}{\#184629}                                                                                               \\ \hline
    Bug \#4  & JavascriptCore                                                  & New                                                                               & \href{https://bugs.webkit.org/show\_bug.cgi?id=184749}{\#184749}                                                                                               \\ \hline
    Bug \#5  & Chakra                                                          & Confirmed                                                                         & \href{https://github.com/Microsoft/ChakraCore/issues/5033}{\#5033}                                                                                           \\ \hline
    Bug \#6  & Chakra                                                          & Fixed                                                                             & \href{https://github.com/Microsoft/ChakraCore/issues/5038}{\#5038}                                                                                           \\ \hline
    Bug \#7  & Chakra                                                          & Confirmed                                                                         & \href{https://github.com/Microsoft/ChakraCore/issues/5065}{\#5065}                                                                                           \\ \hline
    Bug \#8  & \begin{tabular}[c]{@{}c@{}}Chakra\\ JavascriptCore\end{tabular} & \begin{tabular}[c]{@{}c@{}}Confirmed (Chakra)\\ New (JavascriptCore)\end{tabular} & \begin{tabular}[c]{@{}l@{}}\href{https://github.com/Microsoft/ChakraCore/issues/5067}{\#5067}\\ \href{https://bugs.webkit.org/show\_bug.cgi?id=185130}{\#185130}\end{tabular} \\ \hline
    Bug \#9  & JavascriptCore                                                  & New                                                                               & \href{https://bugs.webkit.org/show\_bug.cgi?id=185127}{\#185127}                                                                                               \\ \hline
    Bug \#10 & \begin{tabular}[c]{@{}c@{}}Chakra\\ JavascriptCore\end{tabular} & \begin{tabular}[c]{@{}c@{}}Confirmed (Chakra)\\ New (JavascriptCore)\end{tabular} & \begin{tabular}[c]{@{}l@{}}\href{https://github.com/Microsoft/ChakraCore/issues/5076}{\#5076}\\ \href{https://bugs.webkit.org/show\_bug.cgi?id=185156}{\#185156}\end{tabular} \\ \hline
    Bug \#11 & JavascriptCore                                                  & New                                                                               & \href{https://bugs.webkit.org/show\_bug.cgi?id=185197}{\#185197}                                                                                               \\ \hline
    Bug \#12 & JavascriptCore                                                  & New                                                                               & \href{https://bugs.webkit.org/show\_bug.cgi?id=185208}{\#185208}                                                                                               \\ \hline
    Bug \#13 & Chakra                                                          & New                                                                               & \href{https://github.com/Microsoft/ChakraCore/issues/5128}{\#5128}                                                                      \\ \hline
    \end{tabular}
    \end{table}

\section{Future Steps}

The list below shows future tasks for this project.

\begin{itemize}
\item Integrate different fuzzers, including AFL~\cite{afl},
  HonggFuzz~\cite{honggfuzz}, LibFuzzer~\cite{libfuzzer}, jsfunfuzz~\cite{jsfunfuzz}, and
  Grammarinator~\cite{grammarinator}.

\item Find balance between generational strategies (, which often
  produce valid but simple inputs) and mutational strategies (, which
  often produce complex but invalid inputs) for fuzzing.

\item Explore different strategies to reduce the number of false
  positives as to facilitate the inspection process. For example,
  mining issue trackers for similar bug reports is a candidate alternative.

\item Use our infrastructure to find bugs on
  WebAssembly~\cite{webassembly}, a new binary format already
  supported by major browsers. In principle, our infrastructure is
  agnostic to the runtime engine used. We are eager to explore other
  sources of bugs, especially in very new engines. 

\end{itemize}  

%% \MAB{Unfortunately the low-hanging fruit for this project was already
%%   picked \cite{patra2016learning}, so I believe you should focus on
%%   taming implementation-specific behavior and/or the testing
%%   infrastructure (i.e. how to organize all those fuzzers and make them
%%   work together). You might also want to mention the possibility of
%%   fuzzing
%%   \href{https://developer.mozilla.org/en-US/docs/WebAssembly}{WebAssembly}
%%   directly: LLVM already has (some) support for targeting WebAssembly,
%%   which means that we'll be able to run, in the near future, pretty
%%   much every language with an LLVM-backed compiler in a browser.}

\section{Data Policy}

The results produced with this research will be made available to the
public and to the research community.  All tools and data sets
developed will be available online, and the software will be released
under an open source license.


\footnotesize
\bibliographystyle{plain}
\bibliography{tmp}

\end{document}


%%  LocalWords:  d'Amorim oes Jornalista Fernandes JS Patra Pradel
%%  LocalWords:  fuzzer runtime WebAssembly's codebases Radamsa JSC
%%  LocalWords:  fuzzers linters JavaScriptCore SpiderMonkey AFL
%%  LocalWords:  HonggFuzz jsfunfuzz Grammarinator WebAssembly
