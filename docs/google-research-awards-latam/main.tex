\documentclass[11pt]{article}
%\documentclass[dvips,11pt]{article}

% \usepackage{natbib}
% \setlength{\bibsep}{0pt plus 0.3ex}
\usepackage{titlesec}
\titlespacing*{\section}{0pt}{0.3\baselineskip}{0.3\baselineskip}
%,nohead,nofoot
%\usepackage[margin=1n,left=1cm,top=1cm,right=1.75cm,bottom=1.5cm]{geometry}
\usepackage[margin=1in,includeheadfoot,left=1cm,top=0.5cm,right=1.5cm,bottom=0.5cm]{geometry}
%\usepackage[margin=1in,includeheadfoot]{geometry}
\usepackage{url}
\usepackage{cite}
\usepackage{lipsum}
\usepackage{booktabs}
\usepackage{tabularx}
\usepackage{wrapfig}
\usepackage{microtype}
\usepackage{hyperref}
\usepackage{listings}
\usepackage{subfigure}
% ADD THE FOLLOWING COUPLE LINES INTO YOUR PREAMBLE
\let\OLDthebibliography\thebibliography
\renewcommand\thebibliography[1]{
  \OLDthebibliography{#1}
  \setlength{\parskip}{0pt}
  \setlength{\itemsep}{0pt plus 0.3ex}
}

\usepackage[pdftex]{graphicx}
\usepackage{url}
\usepackage{color}
\definecolor{darkred}{rgb}{0.5,0,0}
\definecolor{darkgreen}{rgb}{0,0.5,0}
\definecolor{darkblue}{rgb}{0,0,0.5}
\definecolor{darkaquamarine}{rgb}{0.2,0.4,0.3}
%% macros
\newcommand{\ie}{i.e.}
\newcommand{\eg}{e.g.}
\newcommand{\Fix}[1]{\textbf{[[}{\color{red} #1}\textbf{]]}}
\newcommand{\Mar}[1]{\textbf{[[}{\color{blue} #1}\textbf{]]}}
\newcommand{\MAB}[1]{\textbf{[[}{\color{darkgreen} #1}\textbf{]]}}
\newcommand{\Igor}[1]{\textbf{[[}{\color{darkaquamarine} #1}\textbf{]]}}
\newcommand{\Comment}[1]{}

%% \setlength{\oddsidemargin}{0.25in}
%% \setlength{\textwidth}{6.5in}
%% \setlength{\topmargin}{0in}
%% \setlength{\textheight}{8.5in}

%\renewcommand\Affilfont{\fontsize{9}{10.8}\itshape}
% These force using more of the margins that is the default style

\begin{document}

\title{Finding Bugs in JavaScript Engines with Differential Testing}
\author{}
\date{}

% You can leave out "date" and it will be added automatically for today
% You can change the "\today" date to any text you like

\makeatletter
\def\maketitle{%
  \par{\centering\large\textbf{\@title}\normalsize\par}\vspace{3ex}%
  \par{\@author}%
  \par}
\makeatother

\maketitle

\vspace{-2ex}
\small    
\begin{table}[h!] 
  \centering%% Centro de Inform\'atica - CIn \\
  \begin{tabular*}{.85\linewidth}{l}

    Marcelo d'Amorim and Igor Sim\~oes (M.Sc. student)\\
    
    Postal Address: Av. Jornalista Anibal Fernandes, S/N. Cidade
    Universit\'aria, PE, Brazil, 50.740-560 \\
    
    Email addresses: \{damorim, isol2\}@cin.ufpe.br\\

    Phones: +55 (81) 2126-8430,
    ext. 4379 [work], +55 (81) 98800-2010 [mobile]\\

    Affiliation: Federal University of Pernambuco, Department of
    Computer Science\\

    Contact at Google:~Martin Barbella (mbarbella@google.com)

  \end{tabular*}
\end{table}
\normalsize

\vspace{-2ex}
\begin{abstract}
...
\end{abstract}

\section{Problem Space}

%% Its niche is front-end and
%% back-end web development\Comment{, supported several frameworks and
%%   runtimes (\eg{}, Express.js and Node.js)}.

%% In fact, bugs in JS implementations are often reported by
%% the general public, some of which were reported by our team.
%% (confirmed by engine developers; see
%% Section~\ref{sec:results}) using a prototype of our solution (see
%% Section~\ref{sec:design}).

JavaScript (JS) is one of the most popular programming languages of
today~\cite{business-insider,stackify} largely because of its
widespread use on the web. As common in software, the JS specification
changes regularly\footnote{Recently, the specification went through a
  bigger change compared to prior releases, called by the name of
  ECMAScript6 (ES6)~\cite{es6-features}} to accomodate pressing
demands of the community and these changes often entail sensible
changes in engine implementations~\cite{kangax} that could lead to
errors, including regressions.  Test generation can help reduce the
cost of testing, but the lack of executable specifications is an
important obstacle for finding functional errors.

%\vspace{-1ex}
\begin{center}
\fbox{
  \begin{minipage}{13.5cm}
    \centering \textit{The goal of this project is to find bugs in
      implementations of JavaScript engines by leveraging diversity
      across implementations.}
  \end{minipage}
}
\end{center}
%\vspace{-1ex}

The novelty of our proposal is the use of multiple engine
implementations (\eg{}, Google's V8, Microsoft's Chakra, etc.) to
assess input validity and to check output correctness. Differential
testing\cite{McKeeman98differentialtesting} has been applied in a
variety of
contexts\cite{Brumley-etal-ss07,Yang-etal-pldi11,Chen-etal-fse2015,Argyros-etla-ccs16,Chen-etal-pldi16,petsios-etal-sp2017,SivakornAPKJ17},
but it has not been thoroughly explored to find functional bugs in JS
engines. Patra and Pradel ran a small study, published in 2016 as
a tech report~\cite{patra2016learning}, where they found
discrepancies\Fix{confirm if they reported bugs}. The Mozilla
jsfunfuzz\Fix{elaborate a bit}.\Mar{Mateus, check if this sounds
  correct/fair to you and try to elaborate the jsfunfuzz part.}

\Fix{finalize}


\section{Design Space}
\label{sec:design}

\begin{wrapfigure}[12]{r}[0pt]{0.45\textwidth}
\vspace{-5ex}
%\begin{figure}[htbp]
  \centering
  \includegraphics[trim=20 350 200 0,clip,width=0.45\textwidth]{google-awards-workflow}
  \label{fig:workflow}
  \caption{Fuzzing Infrastructure.}
%\end{figure}
\end{wrapfigure}
Fuzzing is a popular technique to generate new test inputs from
existing inputs\Fix{cite} with the typical goal of finding
crashes. Several fuzzing methods have been proposed in the past,
varying with respect to how new inputs are generated (\eg{},
mutation-based\Fix{cite}, grammar-based\Fix{cite}, and
random-based\Fix{cite}) and whether or not coverage information is
taken into account in the fuzzing process. Figure~\ref{fig:workflow}
shows a generic pipeline of a fuzzer, instantiated to our problem. In
our case, JS files are provided as seeded inputs to bootstrap the
process. These files can originate from the test suites of the engine
codebases or from the bug reports in their issue trackers. The fuzzing
infrastrucutre selects randomly one of the inputs to fuzz with a given
tool, say the black-box fuzzer Radamsa~\cite{radamsa}. Each fuzzing
iteration produces a new file. For grammar-agnostic fuzzers, like
Radamsa, the infrastructure needs to detect and discard syntactically
invalid inputs. Semantically invalid inputs are detected dynamically
or with linters. Once the input is found to be valid, the
infrastructure invokes the oracle to check correctness. In case either
all engines consider the test output consistent (\ie{}, all pass or
all fail), the infrastructure discards the given input. Otherwise, it
considers that input as potentially fault-revaling and
interesting. Currently, we classify those alarms in two groups: HI and
LO. The HI group includes those warnings that manifest execution
errors spawned directly by the test (or its close neighborhood)
whereas the LO group includes the rest of the warnings. We found
empirically that \Fix{XX\%} of the warnings analyzed in the HI group
were confirmed (see Section~\ref{sec:results})

\section{Related Work}
% Differential testing with fuzzing can improve the quality of programs
% Yang et al. \cite{yang-2011-finding}

In Patra and Pradel \cite{patra2016learning} was described a new black-box fuzzer tool 
that uses generative and probabilistic models to generate the entries. To evaluate the fuzzer,
Patra and Pradel uses JavaScript and HTML documents to improve the javascript engines
in Firefox and Chrome


Hodovan and Kiss \cite{hodovan-fuzzinator} exposes the Fuzzinator\footnote{https://github.com/renatahodovan/fuzzinator}
a random testing framework that uses distincts fuzzers to generate and ran test cases one-by-one in different
program components. Fuzzinator was evaluated on JerryScript\footnote{http://jerryscript.net} 
engine

%% \Mar{rule of thumb: explain the
%%   contribution/novelty, then results achieved, and then how it relates
%% with our proposal.}
%% \Mar{this is not clear$\rightarrow$}Yang et al. \cite{yang-2011-finding} proposed CSmith\footnote{\url{http://embed.cs.utah.edu/csmith/}}, a grammar-fuzzer 
%% of C programs to generate invalid entries\Fix{the goal of a
%%    fuzzer is to generate valid inputs/entries. this is strange}\MAB{the
%%   goal is to generate valid programs without undefined behavior to
%%   facilitate differential testing} and
%% find bugs in several C compilers\Fix{it is strange that you don't mention differential
%%   testing, which is central in CSmith, and the use of LLVM and GCC as
%%   oracles.}.\Mar{this is just listing fuzzers. don't do this. try to
%%   follow rule of thumb above.}Similar fuzzers involving grammar and rules are found in Holler et al. \cite{holler-2012-fuzzing} 
%% that exposes the LangFuzz to generate entries based in code fragments 
%% (language grammar, sample code, test suite) and the tool Mozilla 
%% Funfuzz\footnote{\url{https://github.com/MozillaSecurity/funfuzz}}
%% that implements a fuzzer based on JavaScript language to improve the 
%% testing for SpiderMonkey engine, the interpreter of Mozilla Firefox.
%% \Fix{...}

\MAB{Unfortunately the low-hanging fruit for this project was already
  picked \cite{patra2016learning}, so I believe you should focus on
  taming implementation-specific behavior and/or the testing
  infrastructure (i.e. how to organize all those fuzzers and make them
  work together). You might also want to mention the possibility of
  fuzzing
  \href{https://developer.mozilla.org/en-US/docs/WebAssembly}{WebAssembly}
  directly: LLVM already has (some) support for targeting WebAssembly,
  which means that we'll be able to run, in the near future, pretty
  much every language with an LLVM-backed compiler in a browser.}

\section{Current Results}
\label{sec:results}

\Igor{
  Nowadays, we used six javascript engines suites with approximately 2K of javascript files 
  that could be fuzzed. Currently, using our infrastructure we reported \Fix{13}
  bugs so far. The Table~\ref{tab:bugs} exposes all bugs found and reported by our team.
  At this moment, the status are: \Fix{6} issues was tagged as new, \Fix{4} was confirmed
  and \Fix{2} of them was merged and closed. Unfortunally, \Fix{1} of them was rejected
  due the engine does not follows the specification and dozens of others warnings already been
  reported previously.

  The Figure~\ref{fig:bug-chakra} exposes a fuzzed file generated by radamsa in our environment, it was observed
  that the fuzzer replace the empty string at Line 3 and replace it with the hexadecimal value "\textbackslash x00".
  The bug manifests in Line 7, according to ES6 specification on Mozilla dev website
  \footnote{Unary Plus \url{https://developer.mozilla.org/en-US/docs/Web/JavaScript/Reference/Operators/Arithmetic_Operators}},
  the unary plus converts the variable to an integer if it is a valid value
  otherwise returns NaN. The Figure~\ref{fig:pattern} shows the log output from our infrastructure,
  see that we set automatically this testcase as high priority because it is a failure by assertion.
  The differential test across all engines, shows a unexpected behaviour of ChakraCore engine that passes in this test case.
  It was necessary a manual inspection to check if it is a real bug, we observed that 
  ChakraCore indentify "\textbackslash x00" as a null terminator instead of invalid parser object making
  the unary plus converts null to 0.
}
\Fix{nao estou conseguindo mesclar dois listings com subfigure, por favor alguem tenta fazer isso por mim.}
\Igor{Coloco outro caso de bug reportado?}

\lstdefinelanguage{JavaScript}{
  keywords={break, case, catch, continue, debugger, default, delete, do, else, finally, for, function, if, in, instanceof, new, return, switch, this, throw, try, typeof, var, void, while, with},
  morecomment=[l]{//},
  morecomment=[s]{/*}{*/},
  morestring=[b]',
  morestring=[b]",
  sensitive=true
}

\begin{figure}[htbp]
    \centering
    \begin{subfigure}[b]{0.4}
        \centering
        \scriptsize
        \lstset{
            language=JavaScript,
            keywords={function, return},
            escapeinside={@}{@},
            numbers=left,xleftmargin=1em,frame=single,framexleftmargin=0.5em,
            basicstyle=\ttfamily\scriptsize, boxpos=c,
            numberstyle=\tiny,
            linewidth=6cm
        }
\begin{lstlisting} 
var a = {
    valueOf: function () {
-    return ""
+    return "\x00"
    }
}
assert(+a === 0)
\end{lstlisting}
        \normalsize
        \caption{
            \label{fig:bug-chakra}Fuzzed file shows an unexpected behaviour on ChakraCore
        }        
    \end{subfigure}
    \begin{subfigure}[b]{0.4}
\begin{lstlisting}[basicstyle=\ttfamily]
Priority: HIGH
Pattern:
-----JavaScriptCore
Error: Test failed
-----Chakra
-----SpiderMonkey
Error: Test failed
-----v8
Error: Test failed
\end{lstlisting}
\caption{\label{fig:pattern}\Fix{JSFuzz} output log}
\end{subfigure}
\end{figure}
  

\begin{table}[]
    \centering
    \caption{Bugs found and reported by \Fix{JSFuzz} team}
    \label{tab:bugs}
    \begin{tabular}{|c|c|c|c|}
    \hline
    Issue    & Engine                                                          & Status                                                                            & \multicolumn{1}{c|}{Url}                                                                                                                      \\ \hline
    Bug \#1  & Chakra                                                          & Fixed                                                                             & \href{https://github.com/Microsoft/ChakraCore/issues/4978}{\#4978}                                                                                           \\ \hline
    Bug \#2  & Chakra                                                          & Rejected                                                                          & \href{https://github.com/Microsoft/ChakraCore/issues/4979}{\#4979}                                                                              \\ \hline
    Bug \#3  & JavascriptCore                                                  & New                                                                               & \href{https://bugs.webkit.org/show\_bug.cgi?id=184629}{\#184629}                                                                                               \\ \hline
    Bug \#4  & JavascriptCore                                                  & New                                                                               & \href{https://bugs.webkit.org/show\_bug.cgi?id=184749}{\#184749}                                                                                               \\ \hline
    Bug \#5  & Chakra                                                          & Confirmed                                                                         & \href{https://github.com/Microsoft/ChakraCore/issues/5033}{\#5033}                                                                                           \\ \hline
    Bug \#6  & Chakra                                                          & Fixed                                                                             & \href{https://github.com/Microsoft/ChakraCore/issues/5038}{\#5038}                                                                                           \\ \hline
    Bug \#7  & Chakra                                                          & Confirmed                                                                         & \href{https://github.com/Microsoft/ChakraCore/issues/5065}{\#5065}                                                                                           \\ \hline
    Bug \#8  & \begin{tabular}[c]{@{}c@{}}Chakra\\ JavascriptCore\end{tabular} & \begin{tabular}[c]{@{}c@{}}Confirmed (Chakra)\\ New (JavascriptCore)\end{tabular} & \begin{tabular}[c]{@{}l@{}}\href{https://github.com/Microsoft/ChakraCore/issues/5067}{\#5067}\\ \href{https://bugs.webkit.org/show\_bug.cgi?id=185130}{\#185130}\end{tabular} \\ \hline
    Bug \#9  & JavascriptCore                                                  & New                                                                               & \href{https://bugs.webkit.org/show\_bug.cgi?id=185127}{\#185127}                                                                                               \\ \hline
    Bug \#10 & \begin{tabular}[c]{@{}c@{}}Chakra\\ JavascriptCore\end{tabular} & \begin{tabular}[c]{@{}c@{}}Confirmed (Chakra)\\ New (JavascriptCore)\end{tabular} & \begin{tabular}[c]{@{}l@{}}\href{https://github.com/Microsoft/ChakraCore/issues/5076}{\#5076}\\ \href{https://bugs.webkit.org/show\_bug.cgi?id=185156}{\#185156}\end{tabular} \\ \hline
    Bug \#11 & JavascriptCore                                                  & New                                                                               & \href{https://bugs.webkit.org/show\_bug.cgi?id=185197}{\#185197}                                                                                               \\ \hline
    Bug \#12 & JavascriptCore                                                  & New                                                                               & \href{https://bugs.webkit.org/show\_bug.cgi?id=185208}{\#185208}                                                                                               \\ \hline
    Bug \#13 & Chakra                                                          & New                                                                               & \href{https://github.com/Microsoft/ChakraCore/issues/5128}{\#5128}                                                                      \\ \hline
    \end{tabular}
    \end{table}

\section{Data Policy}

The results produced with this research will be made available to the
public and to the research community.  All tools and data sets
developed will be available online, and the software will be released
under an open source license.


\footnotesize
\bibliographystyle{plain}
\bibliography{tmp}

\end{document}

