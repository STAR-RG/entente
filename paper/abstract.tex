JavaScript is a very popular programming language today. Several
engine implementations compete for market dominance. Although a
specification document and an conformance test suite exist to guide
engine development, bugs occur and have important practical
consequences. One reason for the difficulty in implementing correct
engines stems from the fact that the specification is intentionally
incomplete and evolves frequently.

This paper reports on a study we ran to evaluate the importance of
diversity to find functional bugs in JavaScript engines. For that, we used two
simple diversity-aware techniques--test transplantation and cross-engine
differential testing. The first technique\Comment{Test transplantation} evaluates the effects of
running test suites of a given engine in another engine. The second technique\Comment{Cross-engine
differential testing} evaluates the effects of fuzzing existing inputs
and then comparing the output produced by different engines with a
differential oracle.

We considered engines from four major players in our
experiments--Apple, Google, Microsoft, and Mozilla. Our results
indicate that both techniques revealed several bugs, most of which
confirmed by developers. Test transplantation
revealed \noBugsTransplantation{} bugs
(\noBugsTransplantationConfirmed{} confirmed) and differential testing
revealed \noBugsDifferentialTesting{} bugs
(\noBugsDifferentialTestingConfirmed{}). Furthermore, results indicate
that most of these bugs affected two engines--Apple's
\jsc{} (\percJSC{}) and Microsoft's \chakra{} (\percChakra{}); we found
only \Fix{one} bug in Google \veight{} and none in Mozilla's
\smonkey{}. Although more research needs to be done to optimizie 
warning triaging, our results show that exploring diversity is a
valuable help to find bugs in JavaScript engines.
