\textbf{Context.} 
JavaScript is a very popular programming language today with several
implementations competing for market dominance. Although a
specification document and a conformance test suite exist to guide
engine development, bugs occur and have important practical
consequences. Implementing correct
engines is challenging because the spec is intentionally
incomplete and evolves frequently.
\textbf{Objective.} 
This paper investigates the use of test transplantation and differential testing
for revealing functional bugs in JavaScript engines.
The former technique runs test suites of a given engine on another engine. 
The latter technique fuzzes existing inputs 
and then compares the output produced by different engines with a differential oracle.
\textbf{Method.} 
We conducted experiments with engines from four major players
--Apple, Google, Microsoft, and Mozilla--to assess the effectiveness of 
test transplantation and differential testing.
\textbf{Results.} 
Our results indicate that both techniques revealed several bugs, many of which
confirmed by developers. We reported \noBugsTransplantation{} bugs
with test transplantation (\noBugsTransplantationConfirmed{} of these bugs
confirmed and \noBugsTransplantationFixed{} fixed) and reported
\noBugsDifferentialTesting{} bugs with differential testing
(\noBugsDifferentialTestingConfirmed{} of these confirmed
and \noBugsDifferentialTestingFixed{} fixed). Results indicate that
most of these bugs affected two engines--Apple's
\jsc{} and Microsoft's \chakra{}. 
To summarize, our results show that
test transplantation and differential testing
are easy to apply and very effective in
finding bugs in complex software, such as JavaScript engines.
