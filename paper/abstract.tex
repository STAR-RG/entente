JavaScript is a popular programming language today. Several engine
implementations compete for market dominance. Although a specification
and a conformance test suite exist to guide engine development, bugs
occur and have important consequences given the language
popularity. One reason for the difficulty in implementing correct
engines stems from the fact that the specification is incomplete and
evolves frequently.

This paper reports on a study we ran to evaluate the importance of
diversity in finding bugs on this domain. For that, we explored two
simple diversity-aware methods---test transplantation and cross-engine
differential testing. Test transplantation evaluates the effects of
running test suites of a given engine in another engine. Cross-engine
differential testing evaluates the effects of fuzzing existing inputs
and then comparing the output produced by different engines with a
differential oracle.

We considered engines from four major players in our
experiments--Apple, Google, Microsoft, and Mozilla. Our results
indicate that both techniques revealed several bugs, most of which
confirmed by developers. Test transplantation
revealed \noBugsTransplantation{} bugs
(\noBugsTransplantationConfirmed{} confirmed) and differential testing
revealed \noBugsDifferentialTesting{} bugs
(\noBugsDifferentialTestingConfirmed{}). Furthermore, results indicate
that most of these bugs affected only two engines--Apple's
\jsc{} (\percJSC{}) and Microsoft's \chakra{} (\percChakra{}); we found
only one bug in Google \veight{} and none in Mozilla's
\smonkey{}. Although our experience indicates that more research needs
be done to automate parts of the inspection process, our results show
that exploring diversity is a valuable help to find bugs in JavaScript
engines.
