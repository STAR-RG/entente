JavaScript is a very popular programming language today with several
engine implementations competing for market dominance. Although a
specification document and a conformance test suite exist to guide
engine development, bugs occur and have important practical
consequences. Implementing correct
engines is challenging because the specification is intentionally
incomplete and evolves frequently.

This paper reports on a study we ran to evaluate the importance of
diversity to find functional bugs in JavaScript engines. For that, we
use two simple diversity-aware techniques---test transplantation and
cross-engine differential testing. The first technique runs test
suites of a given engine in another engine. The second technique
fuzzes existing inputs and then compares the output produced by
different engines with a differential oracle.

We considered engines from four major players in our
experiments--Apple, Google, Microsoft, and Mozilla. Our results
indicate that both techniques revealed several bugs, many of which
confirmed by developers. We reported \noBugsTransplantation{} bugs
with test transplantation (with \noBugsTransplantationConfirmed{} bugs
confirmed and \noBugsTransplantationFixed{} fixed) and reported
\noBugsDifferentialTesting{} bugs with differential testing
(with \noBugsDifferentialTestingConfirmed{} confirmed
and \noBugsDifferentialTestingFixed{} fixed). Furthermore, results
indicate that most of these bugs affected two engines--Apple's
\jsc{} (\percJSC{}) and Microsoft's \chakra{} (\percChakra{}); we found
only \Fix{one} bug in Google \veight{} and none in Mozilla's
\smonkey{}. Although more research needs to be done to optimizie 
warning triaging, our results show that exploring diversity is a
valuable help to find bugs in JavaScript engines.
