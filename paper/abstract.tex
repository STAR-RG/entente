\textbf{Context.} JavaScript is a very popular programming language today with several
implementations competing for market dominance. Although a
specification document and a conformance test suite exist to guide
engine development, bugs occur and have important practical
consequences. Implementing correct
engines is challenging because the spec is intentionally
incomplete and evolves frequently.
\textbf{Objective.} This paper evaluates the importance of
diversity to find functional bugs in JavaScript engines. \textbf{Method.} For that, we
explored two existing diversity-aware techniques---test transplantation and
cross-engine differential testing. The first technique runs test
suites of a given engine in another engine. The second technique
fuzzes existing inputs and then compares the output produced by
different engines with a differential oracle.
\textbf{Results.} We considered engines from four major players in our
experiments--Apple, Google, Microsoft, and Mozilla. Our results
indicate that both techniques revealed several bugs, many of which
confirmed by developers. We reported \noBugsTransplantation{} bugs
with test transplantation (\noBugsTransplantationConfirmed{} of these bugs
confirmed and \noBugsTransplantationFixed{} fixed) and reported
\noBugsDifferentialTesting{} bugs with differential testing
(\noBugsDifferentialTestingConfirmed{} of these confirmed
and \noBugsDifferentialTestingFixed{} fixed). Results indicate that
most of these bugs affected two engines--Apple's
\jsc{} and Microsoft's \chakra{}. To summarize, our results show that
diversity-aware techniques are easy to apply and very effective in
finding bugs in complex software, such as JavaScript engines.
