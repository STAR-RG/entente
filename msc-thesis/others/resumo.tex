
% resumo em inglês
\begin{resumo}[Abstract]
 \begin{otherlanguage*}{english}

\noindent
JavaScript is a very popular programming language today with several
implementations competing for market dominance. Although a specification
document and a conformance test suite exist to guide engine development,
bugs occur and have important practical consequences.
This work evaluates the importance of diversity to find functional bugs in
JavaScript engines. For that, we explored two existing diversity-aware
techniques—test transplantation and cross-engine differential testing.
The first technique runs test suites of a given engine in another engine.
The second technique fuzzes existing inputs and then compares the output
produced by different engines with a differential oracle. We considered
engines from four major players in our experiments–V8, SpiderMonkey,
ChakraCore, and JavaScriptCore. The results indicate that both techniques
revealed several bugs, many of which confirmed by developers. Overall,
we reported 49 bugs in this study. Of which, 33 were confirmed by
developers and 22 were fixed. To sum, our results show that diversity-aware
techniques are easy to apply and very effective in finding bugs in
complex software, such as JavaScript engines.
% \lipsum[5]

	\vspace{\onelineskip}
	\noindent 
	\textbf{Key-words}: Diversity. Test Transplantation. Differential Testing. JavaScript.
	\end{otherlanguage*}
\end{resumo}

% resumo em português
\begin{resumo}[Resumo]
	\noindent %- o resumo deve ter apenas 1 parágrafo e sem recuo de texto na primeira linha, essa tag remove o recuo. Não pode haver quebra de linha.
	Atualmente, o JavaScript é uma linguagem de programação muito popular,
	com várias implementações competindo pelo domínio do mercado.
	Embora exista um documento de especificação e um conjunto de testes 
	de conformidade para orientar o desenvolvimento do motor (do inglês, \textit{engine}),
	bugs ocorrem e têm importantes consequências práticas.
	Este trabalho avalia a importância da diversidade para encontrar erros 
	funcionais nos motores JavaScript. Para isso, exploramos duas técnicas existentes 
	sensíveis à diversidade - teste de transplante e teste diferencial entre motores.
	A primeira técnica executa suítes de teste de um determinado mecanismo em outro
	mecanismo. A segunda técnica aplica fuzzing nas entradas de teste e depois
	compara o resultado produzido em diferentes motores através de um oráculo diferencial.
	Consideramos os quatro principais motores da atualidade em nossos
	experimentos - V8, SpiderMonkey, ChakraCore e JavaScriptCore.
	Os resultados indicam que ambas as técnicas revelaram vários bugs,
	muitos dos quais já foram confirmados pelos desenvolvedores.
	No geral, relatamos 49 bugs neste estudo. Dos quais, 33 foram confirmados 
	pelos desenvolvedores e 22 foram corrigidos.
	Em resumo, nossos resultados mostram que as técnicas sensíveis à
	diversidade são fáceis de aplicar e são muito eficazes para encontrar
	bugs em softwares complexos, como motores JavaScript.
	
  % \lipsum[5]
   \vspace{\onelineskip}
   \noindent
   \textbf{Palavras-chaves}: Diversidade. Teste de Transplante. Teste Diferencial. JavaScript.
\end{resumo}