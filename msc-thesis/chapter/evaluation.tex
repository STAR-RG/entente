\chapter{Evaluation}

\section{Results}
\label{sec:results}

The goal of this work is to assess ability of techniques that
leverage diversity to find functional bugs
in \javascript\ engines. Based on that, we pose three questions:
\begin{description}[leftmargin=.5in]
\item[RQ1.] How conformant are the engines to the Test262 suite?
\item[RQ2.] How effective is test transplantation to find bugs?
\item[RQ3.] How effective is cross-engine differential testing to find bugs?  
\end{description}

The first question focuses on the conformance of our selected engines
to the official Test262 suite~\cite{ecma262-conformance-suite}
(Section~\ref{sec:stability}). In the limit, bugs would have low
relevance if the engines are too unreliable. The second question
focuses on the effectiveness of test transplantation
(Section~\ref{sec:transplantation}). The rationale for using inputs
from different engines is that developers consider different goals
when writing tests---suites written for a given engine may cover
scenarios not covered by a different engine. The third question
evaluates the effectiveness of cross-engine differential testing to
find bugs (Section~\ref{sec:cross-engine-diff-testing-results}). The
rationale for this question is that fuzzing inputs may explore
scenarios not well-tested by at least one of the engines.

\subsection{Answering RQ1 (Conformance)}
\label{sec:stability}

The ECMA Test262~\cite{ecma262-conformance-suite} test suite serves to
check conformance of engines to the \js\ standard. It is
acceptable to release engines fulfilling the specification only
partially~\cite{kangax}. We expect that the pass rate on this suite
provide some indication of the engine's maturity. In the limit, it is
not desirable to flood bug reports on engines at early stages of
development. 
\Igor{R1 criticou o pq este experimento. Não faz muito sentido deixar
isso, já que nao houve quase nenhum progresso durante a semana.
Ou podemos deixar executando varias vezes por dia, durante uns 15 dias ou 1 mes
para monitorar o progresso.
}
For this experiment, we ran the suite once a day for seven consecutive
days and averaged the passing ratios. Table~\ref{tab:test262} shows
the average number of passing tests over this period. The variance of
results was negligible; for that reason, we omitted standard
deviations. We noticed that all four engines but \chakra{} used some
variant of the Test262 suite as part of their regression process, but
we used the same version in this
experiment~\cite{ecma262-conformance-suite}. 

\begin{table}
  %\vspace{-4ex}
  \centering
  \caption{\label{tab:test262}Percentage of passing tests on
    the Test262 conformance suite.}
  %\small
  \begin{tabular}{rr}
    \toprule
    engine & \% passing \\
    \midrule
    \jsc{} & \percentSuiteTestJSC{}\\
    \veight{} & \percentSuiteTestVeight{} \\
    \chakra{} & \percentSuiteTestChakra{} \\    
    \smonkey{} & \percentSuiteTestSM{} \\
    \bottomrule
  \end{tabular}
  %\normalsize
\end{table}

Results show that there are still many unsupported scenarios as can be observed from the
percentages in the table. The number of passing tests is high and
similar for \jsc{}, \veight{}, and \smonkey\ whereas \chakra\ performs
worse compared to the other engines. Note also that \chakra\ is both
the engine that has the lowest passing ratio in this test suite and
the one we were able to find more bugs (as per
Figure~\ref{fig:summary}). Although it is plausible to find
correlation between the passing ratios and reliability as measured by
the number of bugs found, we do not imply causality. It is important
to note that failures in this conformance test suite indicates missing
features as opposed to bugs.

\begin{center}
  \fbox{
    \begin{minipage}{8cm}
      \textit{Summary:}~All engines seem to adhere well to the
      \js\ standard. Except for \chakra, the passing ratio of all
      engines is above 90\%.
      \end{minipage}
    }
\end{center}


\subsection{Answering RQ2 (Test Transplantation)}
\label{sec:transplantation}

This section reports results of test transplantation. More
specifically, we analyzed the failures observed when running a test
suite original from a given engine in another engine. Intuitively, we
want to assess how effective is the idea of cross-fertilization of
testing knowledge among \js\ developers.

%\vspace{0.5ex}
\subsubsection{Methodology}
\label{sec:methodology}
\Igor{manter isso?}
\Fix{In this experiment, a developer with experience in \js\ analyzed each
test failure, affecting a particular engine, and classified that
failure as potentially fault-revealing or not. The authors supervised
the classification process to validate correctness. For the
potentially fault-revealing cases, one of the authors inspected the
scenario and, if agreed on the classification, reported the bug to the
issue tracker of the affected engine.}
% In this experiment, we analyzed each test failure,
% affecting a particular engine, and classified that
% failure as potentially fault-revealing or not. This methodology
% was supervised the classification process to validate correctness. For the
% potentially fault-revealing cases, if agreed on the classification, we 
% reported the bug to the issue tracker of the affected engine.


\begin{table}[t]
  \small
  \centering
  \caption{\label{tab:cross-testing}Number of failures with
    Test Transplantation.}
  \renewcommand*{\arraystretch}{0.9}
  \begin{tabular}{crrrr}
    \toprule
    test suite\textbackslash{}engine & \jsc{} & \veight{} & \smonkey{} & \chakra{}\\
    \midrule
    % \Fix{Versions (03.08)} & 234555 & 7.0.158 & 62.0b14 & 1.10.1 \\
    % \Comment{
    %   Lembrar dos testes que os testes da propria engine falham:
    %   V8 0 
    %   JSC 2 
    %   Spidermonkey 58
    % }
    \jsc{} & - & 10 & 10 & 59   \\
    \veight{} & 41 & - & 3 & 5  \\
%    \chakra{} & - & - & - & -  \\
    \smonkey{} & 218 & 107 & - & 281 \\
    Duktape & 0 & 4 & 4 & 1   \\
    \jerry{} & 23 & 25 & 22 & 23   \\
    JSI & 0 & 0 & 0 & 0   \\ 
   Tiny-js & 0 & 0 & 0 & 0  \\
    \midrule
   \textbf{total} & 282 & 146 & 39 & 369 \\
    \bottomrule 
  \end{tabular}
  \vspace{-3ex}
\end{table}

\subsubsection{Results}
\label{sec:results}

Table~\ref{tab:cross-testing} shows the number of failures observed
for each pair of test suite and engine. The first column shows the
test suites and the first row shows the engines that run those
tests. We use a dash (``-'') to indicate that we did not consider the
combinations that associate a test suite with its parent engine;
failures in those cases would either indicate regressions or flaky
tests as opposed to unknown bugs for that engine. As explained in
Section~\ref{sec:seeds}, we used in this experiment the
\totalTestFilesForTestTransplantation{} tests included in the
rectangle area under column ``type-in-all'' from
Table~\ref{tab:test-suites}. Running those tests we observed a total
of \failuresTestTrans{} failures manifested across
\failuresTestTransDistictFiles{} distinct files
(\failuresTestTransPercent{} of total).  Table~\ref{tab:cross-testing}
shows that \smonkey\ was the engine that failed the least whereas
\chakra\ was the engine that failed the most. The \smonkey\ test suite
also revealed more failures than any other, as expected, given
that it is the suite with more tests (see
Table~\ref{tab:test-suites}).


%%  As another example, some tests from \Fix{Y} use the
%% implementation-dependent description message\Mar{what is the name of
%%   the property?} encapsulated in an \CodeIn{Error} object in
%% assertions. As such, those tests fail in all engines but
%% \Fix{Y}. 

\sloppy The sources of FP found in this experiment are as
follows: \textbf{Undefined Behavior.} FP of this kind are
manifested when tests cover implementation-dependent behavior, as
defined in the ECMA262 specification~\cite{ecmas262-spec}. For
example, one of the tests from \jerry\ uses the function
\CodeIn{Number.toPrecision([precision])}, which translates a number to
a string, considering a given number of significant digits. The
floating-point approximation of the real value is
implementation-dependent, making that test to pass only in
\chakra. \textbf{Timeout/OME\footnote{ome is for out of memory
    error.}.} FP of this kind typically manifest when the
engine that runs the test does not optimize the code as the original
engine of the test. As result, the test fails to finish at the
specified time budget or it exceeds the memory budget. For example, a
test case from \jsc{} defines a function with a tail-call
recursion. The test fails in all engines but \jsc{}, which implements
tail-call optimization. \textbf{Not implemented.} FP of
this kind manifest when a test fails because it covers a function that
is part of the official spec, but is not implemented in the target
engine yet. For example, at the time of writing, \chakra{} did not
implement by default various properties from the \mcodeid{Symbol}
object. These properties are only available activating the ES6
experimental mode with the flag \CodeIn{-ES6Experimental}.
\textbf{Non-Standard Element.} These cases manifest when a function or
an object property is undefined in the execution engine but we were
unable to capture that by looking for error types like
\CodeIn{ReferenceError} and \CodeIn{TypeError}.
% \Comment{ For example,
%   we found cases where the test (or its harness) fails in an assertion
%   because of an undefined property.}
\textbf{Other.} This category
includes other sources of FP. For example, it includes
the cases where the test was valid for some previous version of the
spec but is no longer valid for the current spec.

\begin{table}
  \centering
  \caption{\label{fig:falsepositives}\label{fig:truepositives}\label{fig:piecharts-transplantation}Distribution
    of FP and TP.}
  %\setlength{\tabcolsep}{3pt}      
  \renewcommand*{\arraystretch}{0.9}
  \begin{tabular}{ccr}
    \toprule
    & source &  \#\\
    \midrule
    \multirow{5}{*}{FP} & Undefined Behavior & \noTransUndefined{} \\
    & Timeout/OME & \noTransTimeout{} \\
    & Not Implemented & \noTransNotImplemented{} \\
    & Non-Standard Element & \noTransNonStandard{} \\
    & Other & \noTransOther{} \\
    \midrule
    \multirow{2}{*}{TP} & Duplicate & \noTransTPDuplicated{} \\
    & Bug & \noTransTPBugs{} \\
    \bottomrule
  \end{tabular}
\end{table}

Table~\ref{fig:piecharts-transplantation} shows the distribution of
\gls{fp} and \gls{tp}. The sum of the numbers
in this table correspond to the number of files that manifested
failures, \ie{}, \failuresTestTransDistictFiles{}. Considering false
positives, ``Undefined Behavior'' was the most predominant
source. Considering TP, we found a reasonable number of
duplicate reports, but not high enough to justify attempting to
automate the detection of duplicates.

\begin{table}[t!]
  %\small
%  \setlength{\tabcolsep}{5pt}
  \renewcommand{\arraystretch}{0.95}
  %  \vspace{-3ex}
    %  \scriptsize
      \centering
      \caption{List of bugs reports from Test Transplantation.}
      \label{tab:test-transplantation-bugs}
    
      \begin{tabular}{rcccccc}
        \toprule Issue\# & Date & Engine  & Status & Url & Severity & Suite \\
        \midrule
       1  & 4/18  & \jsc{}  & New  & \href{https://bugs.webkit.org/show\_bug.cgi?id=184749}{\#184749} & - & \jerry{} \\
       2  & 4/23 & \chakra{}  & \textbf{Confirmed}  & \href{\repoCH/issues/5033}{\#5033} & 2 & \smonkey{} \\   
       3  & 4/29 & \chakra{}  & \textbf{Fixed}   & \href{\repoCH/issues/5065}{\#5065} & 2 & \smonkey{} \\
%      \midrule
       4  & \multirow{2}{*}{4/29} & \chakra{} & \textbf{Confirmed} & \href{\repoCH/issues/5067}{\#5067} & 2 & \smonkey{} \\
       5  &  & \jsc{} & New & \href{https://bugs.webkit.org/show\_bug.cgi?id=185130}{\#185130} &  -  & \smonkey{}\\
%      \midrule
       6 & 5/02  & \jsc{} & New  & \href{https://bugs.webkit.org/show\_bug.cgi?id=185208}{\#185208} & - & \smonkey{} \\
       7 & 5/02  & \jsc{} & \textbf{Confirmed}  & \href{https://bugs.webkit.org/show_bug.cgi?id=185211}{\#185211} & 2 & \smonkey{}\\ % webkit
       %  8 & 5/02  & \chakra{} & \textbf{Confirmed} & \href{\repoCH/issues/5187}{\#5087}}} & 3 & \smonkey{}\\
       8 & 5/17  & \chakra{} & \textbf{Confirmed} & \href{\repoCH/issues/5187}{\#5187} & 2 & \jsc{}\\
       9 & 5/21  & \chakra{} & \textbf{Fixed} & \href{\repoCH/issues/5203}{\#5203} & 2 & \smonkey{}\\
       10 & 6/28  & \chakra{} & \textbf{Confirmed}  & \href{\repoCH/issues/5388}{\#5388} & 2 & \jsc{}\\
       11 & 7/10  & \chakra{} & \textbf{Confirmed} & \href{\repoCH/issues/5442}{\#5442} & 2 & \jerry{}\\ % duktape
       12 & 7/18  & \chakra{} & \textbf{Fixed} & \href{\repoCH/issues/5478}{\#5478} & 2 & \smonkey{}\\
       13 & 7/18  & \jsc{} & New & \href{https://bugs.webkit.org/show_bug.cgi?id=187777}{\#187777} & - & \jerry{}\\
       14 & 7/18  & \chakra{} & \textbf{Fixed} & \href{\repoCH/issues/5549}{\#5549} & 2 & \jerry{}\\ % test262
       15 & 8/07  & \chakra{} & \textbf{Fixed} & \href{\repoCH/issues/5576}{\#5576} & 3 & \jerry{}\\
       16 & 8/07  & \jsc{} & \textbf{Fixed} & \href{https://bugs.webkit.org/show_bug.cgi?id=188378}{\#188378} & 2 & \jerry{}\\
       17 & 8/07  & \chakra{} & \textbf{Confirmed} & \href{\repoCH/issues/5579}{\#5579} & 3 & \jerry{}\\
       18 & 8/07  & \jsc{} & \textbf{Fixed} & \href{https://bugs.webkit.org/show_bug.cgi?id=188382}{\#188382} & 2 & \jerry{}\\
       19 & 8/08  & \veight{} & \textbf{Fixed} & \href{https://bugs.chromium.org/p/v8/issues/detail?id=8033}{\#8033} & 3 & \jerry{}\\
       20 & 8/08  & \jsc{} & New & \href{https://bugs.webkit.org/show_bug.cgi?id=188407}{\#188407} & - & \jerry{}\\
       21 & 8/16  & \veight{} & WontFix & \href{https://bugs.chromium.org/p/v8/issues/detail?id=8064}{\#8064} & 2 & Duktape\\
       22 & 8/22  & \chakra{} & \textbf{Fixed} & \href{\repoCH/issues/5621}{\#5621} & 2 & \smonkey{}\\
       23 & 8/22  & \jsc{} & \textbf{Fixed} & \href{https://bugs.webkit.org/show_bug.cgi?id=188874}{\#188874} & 2 & \smonkey{}\\
       24 & \multirow{3}{*}{8/22} & \jsc{} & New & \href{https://bugs.webkit.org/show_bug.cgi?id=188875}{\#188875} & - & \smonkey{}\\
       25 &  & \veight{} & \textbf{Fixed} & \href{https://bugs.chromium.org/p/v8/issues/detail?id=8082}{\#8082} &  2  & \smonkey{}\\
       26 &  & \chakra{} & \textbf{Fixed} & \href{https://github.com/Microsoft/ChakraCore/issues/5624}{\#5624} &  2  & \smonkey{}\\
       27 & 8/23  & \jsc{} & New & \href{https://bugs.webkit.org/show_bug.cgi?id=188877}{\#188877} & - & \smonkey{}\\
       \bottomrule
      \end{tabular}
      \vspace{-2ex}
\end{table}



Table~\ref{tab:test-transplantation-bugs} lists all bugs we found with
test transplantation. The first column shows the identifier we
assigned to the bug,%\Comment{, column ``Date'' shows the date the bug was reported in the format ``m/dd'',}
column ``Engine'' shows the
affected engine, column ``Status'' shows the status of the bug report
at the time of the writing. The status string appears in bold face for
status ``Confirmed'' or higher, \ie{}, ``Assigned'' and ``Fixed''.
Column ``Severity'' shows the severity of confirmed bugs, and,
finally, column ``Suite'' shows the name of the engine that originated
the test.  (We omitted the URLs of the issues to protect anonymity.)
Considering severity levels, we found that \jsc{}~\cite{jsc-severity}
and \smonkey{}~\cite{mozilla-severity} developers use five levels,
whereas \chakra{}~\cite{chakra-severity} and
\veight{}~\cite{v8-severity} developers use only three. As usual, the
smallest the severity value the highest the severity of the bug. We
use a dash (``-'') in place of the severity level for the cases where
the bug report is pending confirmation. Of the
\noBugsTransplantation{} bugs we reported in this experiment,
\noBugsTransplantationConfirmed\ were promoted from the status ``New''
to ``Confirmed''. Of these, \noBugsTransplantationSeverityTwo{} are severity-2 bugs.  Although we did
not find any critical bugs, most of the bugs are seemingly important
as per the categorization given by engineers. Analyzing the issue
tracker of \chakra, we found that severity-1 bugs are indeed
rare. Considering the number of bug reports confirmed by developers,
\chakra{} was the engine with the highest number--\bugsChakra{} (with \bugsChakraFixed{} bugs
fixed). Considering the remaining engines, \veight{} developers
confirmed the \noTransVeightBugsReported{} bugs we reported, fixing 
\noTransVeightBugsFixed{}. Curiously, Google
engineers confirmed the issued bug reports in a few hours. Overall, we
found that development teams of other engines, specially \jsc\, took
much longer to analyze bug reports as can be observed in the
\jsc\ stacked bar from Figure~\ref{fig:stacked-engine}. 
However, the bugs the \jsc{} team confirmed were quickly fixed.

%% Overall, we found these results promising given that precision (\ie{},
%% the ratio of true positives over all reports) needs to take into
%% account the cost of inspecting the failure~\cite{harman-ohearn-2018};
%% it took a week for the developer to inspect the entire set of
%% failures. The precision for \jsc\ was \Fix{X\%} (\ie, XX/270), which
%% seems a low rate, but it took approximately \Fix{x} days to find
%% \Fix{y} real bugs on this engine. Section~\ref{sec:bugs} shows samples
%% of false and true positives.

\begin{center}
  \fbox{
    \begin{minipage}{8cm}
      \textit{Summary:}~Test transplantation was effective at finding
      functional bugs. Although the cost of classifying failures was
      non-negligible, the approach revealed several non-trivial bugs
      in three of the four engines we analyzed.
      \end{minipage}
    }
\end{center}


\subsection{Answering RQ3 (Differential Testing)}
\label{sec:cross-engine-diff-testing-results}

This section reports the results obtained with cross-engine differential
testing. More precisely, we report results obtained by fuzzing test
inputs, running those inputs on different engines, and checking the
outcomes with a differential oracle.

\vspace{0.5ex}
\subsubsection{Methodology}
The experimental methodology we used is as follows. As explained in
Section~\ref{sec:objects:fuzzers}, we used Radamsa~\cite{radamsa} and
QuickFuzz~\cite{grieco2016quickfuzz} for fuzzing. To avoid experimental noise,
we only fuzz test files that pass in all engines--a total of
\totalTestFilesPassInAll\ tests satisfy this restriction. Those tests
appear under the column ``no-fail-in-all'' on
Table~\ref{tab:test-suites}.  We want to avoid the scenario where
fuzzing produces a fault-revealing input based on a test that was
already revealing failures on some engine. This decision facilitates
our inspection task; it helps us establish cause-effect relationship
between fuzzing and the observation of discrepancy.  We configured our
infrastructure (see Figure~\ref{fig:workflow}) to produce 20
well-formed fuzzed files per input file, \ie{}, the number of fuzzing
iterations can exceed the number above as we discard generated files
that are syntactically invalid.

%%  That
%% finding would be important to reduce inspection cost and guide future
%% developments of this approach. 

\vspace{0.5ex} \lbrack{}Exploratory Phase.\rbrack{} For the first
three months of the study, our inspection process was exploratory.  In
this phase, we wanted to learn whether or not black-box fuzzers could
reveal real bugs and how effective was the \hi{}-\lo\ warning
classification.  We expected the number of warnings to increase
dramatically compared to the previous experiment and, if we realized
that the ratio of bugs from \lo\ warnings was rather low, we could
focus our inspection efforts on \hi\ warnings. To run this experiment,
we trained eight students in analyzing the warnings that our
infrastructure produces. The students were enrolled in a
graduate-level testing class. We listed warnings in a spreadsheet and
requested the students to update an ``owner'' column indicating who
was working on it, but we did not enforce a strict order on the
warnings the students should inspect. Recall from
Section~\ref{sec:clusterization} that we clustered \lo\ warnings in
buckets. For that reason, we only listed one \lo\ warning per
representative class/bucket in the spreadsheet. First, we explained,
through examples, the possible sources of false alarms they could find
and then we asked the students to use the following procedure when
finding a suspicious warning. Analyze the parts of the spec related to
the problem and, if still suspicious, look for potential duplicates on
the bug tracker of the affected engine using related keywords. If none
was reported, indicate in the spreadsheet that that warning is
potentially fault-revealing. We encouraged students to use
lithium~\cite{lithium} to minimize long test cases.
% \Comment{; the tests from the Mozilla suite are often longer than others}
A bug report was filed only after one of the authors reviewed the diagnosis.
Each student found at least one bug using this methodology.

\vspace{0.5ex}
\lbrack{}Non-Exploratory Phase.\rbrack{} Results obtained in the
exploratory phase confirmed our expectations that most of the bugs
found during the initial period of investigation were related to
\hi\ warnings. For that reason, we changed our inspection
strategy. This time, only the co-authors inspected the bugs using a
similar discipline as before. However, the set of warnings inspected
and the order of inspection changed. We restricted our analysis to
\hi\ warnings and, aware that we would be unable to analyze each and
every warning reported, we grouped those warnings per engine,
analyzing each group in a round-robin fashion.  At each iteration, we
analyzed \warningsIteration{} warnings in each group. A warning
belongs to the group of a given engine if only that engine manifests
distinct behavior, \ie{}, it produces a distinct output compared to
others. We separated in a distinct group the warnings for which two
engines diverge. The rationale for this methodology was to give
attention to each engine more uniformly, enabling more fair comparison
across engines.

%% Table~\ref{tab:summary-lo}
%% shows statistics of \lo\ warnings obtained when fuzzing inputs with
%% different tools. Column ``\# buckets'' shows the number of buckets
%% associated with ``\lo'' warnings, column ``\# files'' shows the total
%% number of files involved in these warnings, and column ``avg. \# files
%% per bucket'' shows the average number of files per bucket. Note that,
%% although the number of buckets and files differ substantially across
%% fuzzers, the difference of files per bucket was not very high.\Mar{not
%% sure how relevant is this par.}

%% \begin{table}[t]
%%   \small
%%   \centering
%%   \caption{\label{tab:summary-lo}Summary of \lo\ warning reports.}
%%   \begin{tabular}{crrr}
%%     \toprule
%%     & \# buckets & \# files & avg. \# files per bucket \\
%%     \midrule
%%     \radamsa{} & 82 & 319 & 3.8 \\
%%     \quickfuzz{} & 78 & 218 & 2.7 \\
%%     \bottomrule
%%   \end{tabular}
%% \end{table}

\vspace{0.5ex}
\subsubsection{Results}~Table~\ref{tab:summary-hi} shows statistics of \hi\ warnings. The table breaks down \hi\ warning by the affected
engine, \ie, the engine manifesting distinct output among those
analyzed. Column ``+1'' shows the cases where more than one engine
disagree on the output. Note from the totals that the ordering of
engines is consistent with the one observed on
Table~\ref{tab:cross-testing}, with \chakra\ and \jsc\ in first and
second places, respectively, in number of warnings.


\begin{table}[t]
  \small
  \setlength{\tabcolsep}{4.5pt}
  \centering
  \caption{\label{tab:summary-hi}Number of \hi\ warning
    reports per engine.}
  \begin{tabular}{crrrrr}
    \toprule
    fuzzer\textbackslash{}engine & \jsc\ & \veight\ & \chakra & \smonkey & +1\\
    \midrule
    \radamsa{} & 151 & 50 & 331 & 94 & 628 \\ % 137 & 47 & 323 & 21 & 483
    \quickfuzz{} & 83 & 63 & 351 & 21 & 403 \\ % 81 & 62 & 349 & 13 & 369
    \midrule
    \textbf{total} & 234 & 113 & 682 & 115 & 1031 \\ % 218 & 109 & 672 & 34 & 852
    \bottomrule
  \end{tabular}
\end{table}


Table~\ref{tab:false-positives} shows the distribution of false
positives per source. The sources of imprecision are as defined in
Section~\ref{sec:transplantation} with the addition of two new
sources, which we did not observe before. These new sources are
marked with a ``*'' in the table. The source ``Invalid Input''
indicates that the test input violated some part of the
specification. For example, the test indirectly invoked some function
with unexpected arguments; this happens because fuzzing is not
sensitive to function specifications. Consequently, it can replace
valid with invalid inputs. The source ``Error Message Mismatch''
corresponds to the cases where the fuzzer modifies the assertion
expression (\eg{}, some string expression or regular expression).

\begin{table}[t]
  \small
  \centering
  \caption{\label{tab:false-positives}Distribution of FP and TP.}
  \begin{tabular}{ccrr}
    \toprule
    & & \radamsa\ & \quickfuzz\ \\
    \midrule
    \multirow{4}{*}{FP} & Undefined Behavior & 42 & 16 \\ % 26 & 9 
    & Timeout/OME & 30 & 15 \\ % 28 & 13
    & * Invalid Input & 46 & 55 \\ % 39 & 48
    & * Error Message Mismatch & 41 & 12 \\ % 41 & 12
    \midrule
    \multirow{2}{*}{TP} & Duplicate & 36 & 28\\ % 34 & 28
    & Bug & 15 & 7\\
    \bottomrule     
  \end{tabular}
\end{table}

%% \Igor{devemos remover isso da 
%% etapa de fuzzing upcoming feature falha mesmo sem fuzzing...$\rightarrow$
%% The source ``Upcoming Feature''
%% indicates that the test fails because it uses a feature that is
%% available in the original engine, but it is still not standard, and
%% the target engine does not support that feature}. For example,
%% \Fix{provide concrete example}. \Fix{explain error message mismatch}


%% The table is
%% organized in two sections. The top section shows results obtained
%% during the exploratory phase of the study and the bottom section shows
%% the results obtained during the non-exploratory phase.


Table~\ref{tab:bugs} shows the list of bugs we reported. The table
shows the fuzzing tool used
(``Fuzzer''), the \js\ engine affected (``Engine''), the status of the
bug report (``Status''),
% \Comment{ as of Aug. 24, 2018, the Url of the bug report (``Url'')}
the severity of the bug report (``Sev.''), the priority
that we assigned to the warning that revealed the bug (``Priority''),
and the test suite from the original test input (``Suite''). So far,
\noDiffConfirmed{} of the bugs we reported were confirmed, \noDiffFixed{} of which
were fixed. Note that one bug report that we submitted was rejected on
the basis that the offending JS file manifested an incompatibility
across engine implementations that was considered to be acceptable. As
of now, we did not find any new bugs on \smonkey{}; the bugs we found
were duplicates and were not reported. For \veight{}, 
we reported \noDiffVeight{} bugs, 
all of them confirmed, with \noDiffVeightFixed{} fixed.

\begin{table}[t]
  %\small
  \vspace{-3ex}
  \centering
  \caption{List of bugs reports from Differential Testing.}
  \label{tab:bugs}
  \setlength{\tabcolsep}{1.5pt}
  \renewcommand{\arraystretch}{0.95}
  \begin{tabular}{rcccccccc}
    \toprule
    Issue\#  & Date & Fuzzer & Engine  & Status & Url  & Sev. & Priority & Suite \\
    \midrule
    1  & 4/12 & radamsa & \chakra{}   & \textbf{Fixed}  &
    \href{\repoCH/issues/4978}{\#4978}
    & 2 & \textbf{\lo} & \jsc{} \\ 
    2  & 4/12 & radamsa & \chakra{}   & WontFix  &
    \href{\repoCH/issues/4979}{\#4979}
    & - & \hi{} & \jsc{} \\
    3  & 4/14 & radamsa & \jsc{}  & New &
    \href{https://bugs.webkit.org/show\_bug.cgi?id=184629}{\#184629}
     & -  & \hi{} & \jsc{}    \\
    4  & 4/25 & radamsa & \chakra{}  & \textbf{Fixed}     &
    \href{\repoCH/issues/5038}{\#5038}
    & 2 & \hi{} & \jerry{}   \\
    5  & 4/29 & radamsa & \jsc{}  & \textbf{Fixed}  &
    \href{https://bugs.webkit.org/show\_bug.cgi?id=185127}{\#185127}
     & 2  & \hi{}  & \jerry{}\\
    
    6 & 4/30  & radamsa & \chakra{} & \textbf{Confirmed} &
    \href{\repoCH/issues/5076}{\#5076}
    & 2 & \hi{} & TinyJS\\
    
    7                    & 4/30                       &  radamsa   &
    \jsc{} & New &
    \href{https://bugs.webkit.org/show\_bug.cgi?id=185156}{\#185156}
    & - & \hi{} & TinyJS \\


    
    8 & 5/02 & radamsa & \jsc{}  & \textbf{Fixed} &
    \href{https://bugs.webkit.org/show\_bug.cgi?id=185197}{\#185197}
    & 2 & \textbf{\lo} & \smonkey{} \\
    9 & 5/10 & radamsa & \chakra{} & \textbf{Confirmed} &
    \href{\repoCH/issues/5128}{\#5128}
    & 3 & \hi{} & \jerry{} \\
    10 & 5/17 & radamsa & \chakra{} & \textbf{Fixed} &
    \href{\repoCH/issues/5182}{\#5182}
    & 2 & \hi{} & \veight{}\\
    11 & 5/24 & radamsa & \jsc{} & \textbf{Fixed}  &
    \href{https://bugs.webkit.org/show\_bug.cgi?id=185943}{\#185943}
    & 2 & \hi{} & \jsc{}\\
    12 & 6/26 & radamsa & \jsc{} & \textbf{Fixed}  &
    \href{https://bugs.webkit.org/show_bug.cgi?id=187042}{\#187042}
    & 2 & \hi{} & \jerry{}\\
    13 & 7/10 & quickfuzz & \jsc{} & \textbf{Fixed}  &
    \href{https://bugs.webkit.org/show_bug.cgi?id=187520}{\#187520}
    & 2 & \hi{} & \jerry{}\\
    14 & 7/10 & quickfuzz & \chakra{} & \textbf{Confirmed}  &
    \href{\repoCH/issues/5443}{\#5443}
    & 2 & \hi{} & \jerry{}\\
    %    \multicolumn{9}{c}{\emph{Non-Exploratory Phase}} \\



    15 & 8/21  & radamsa & \chakra{} & \textbf{Fixed} &
    \href{\repoCH/issues/5617}{\#5617}
    & 3 & \hi{} & Test262\\    

    16 & 8/21 & radamsa &
    \veight{} & \textbf{Fixed} &
    \href{https://bugs.chromium.org/p/v8/issues/detail?id=8078}{\#8078}
    & 2 & \hi{} & Test262 \\

    
    17 & 8/23 & quickfuzz & \jsc{} & New  &
    \href{https://bugs.webkit.org/show_bug.cgi?id=188899}{\#188899}
    & - & \hi{} & Test262\\
    18 & 8/23 & quickfuzz & \veight{} & \textbf{Confirmed}  &
    \href{https://bugs.chromium.org/p/v8/issues/detail?id=8088}{\#8088}
    & 2 & \hi{} & Test262\\

    19 & 8/24  & quickfuzz & \chakra{} & \textbf{Confirmed} &
    \href{\repoCH/issues/5630}{\#5630}
    & 2 & \hi{} & Test262\\
    
    20 & 8/24 &  quickfuzz & 
    \jsc{} & New &
    \href{https://bugs.webkit.org/show_bug.cgi?id=188920}{\#188920}
    & - & \hi{} & Test262 \\

    21 & 8/24 & quickfuzz & \jsc{}  & New &
    \href{https://bugs.webkit.org/show\_bug.cgi?id=188930}{\#188930}
    & - & \hi{} & Test262\\
    
    22 & 8/21  & radamsa & \chakra{} & \textbf{Confirmed} &
    \href{\repoCH/issues/5968}{\#5968}
    & 3 & \hi{} & \smonkey{}\\
   \bottomrule
  \end{tabular}
\end{table}


%% Note from the table that all bug reports still waiting for
%% confirmation are associated with the \jsc{} engine. A closer look at
%% the \jsc{} issue tracker showed that their triage process is slow
%% relative to \chakra's.\Comment{\Igor{ The bugs reported on \chakra{}'s
%%     bug tracker was defined as confirmed, however the bugs are
%%     included on a milestone for the next release following an internal
%%     priority.  }} 

%% Finally, it is worth noting that \Fix{11 of the 19}
%% JS files that manifested discrepancies were \emph{not} produced with
%% fuzzing (column ``Fuzz''). These are test files from suites of
%% different engines. This observation emphasizes the importance of
%% continuously collecting test suites from multiple sources; today, we
%% use test suites from seven different open source engines, including a
%% total of 30K test files.

\begin{center}
  \fbox{
    \begin{minipage}{8cm}
      \textit{Summary:}~Cross-engine differential testing was
      effective at finding \js\ engines bugs, several of which have
      been fixed already.
      \end{minipage}
    }
\end{center}

\noindent
\textbf{Data Availability.}~The data, including the tests, warning
reports, and diagnostic outcomes, is publicly available from a
preserved repository \dataRepo.

\section{Discussion}
\label{sec:bugs}


\subsection{Example Bug Reports}

This section discusses a sample of bugs reports. The selection
criteria we used was: (i) to cover all engines we found bugs--\chakra,
\jsc, and \veight, (ii) to cover each technique--test transplantation
and differential testing (with \radamsa\ and with \quickfuzz), (iii)
to cover a case of rejected bug report, and (iv) to use short tests
(for space).

%% \CodeIn{\{var a =
%%   \{valueOf:~function()\{} \CodeIn{return ``\textbackslash{}x00''\}\}
%%   assert(+a === 0)\}}

\sloppy
\vspace{1ex}\noindent\textbf{Issue \#4, Table~\ref{tab:bugs}.} The
code snippet below shows a test that reveals a bug in \chakra.

\begin{figure}[h!]
  \vspace{-1ex}
  \centering
  \scriptsize
  \lstset{escapeinside={@}{@},
%    numbers=left,xleftmargin=1em,frame=single,framexleftmargin=0.5em,
    basicstyle=\ttfamily\scriptsize, boxpos=c,
    numberstyle=\tiny,
    morekeywords={assertEq, var, yield, in, function, 
    typeof, return, throw, new, Error, if},
  }
  \begin{lstlisting}
 { var a = {valueOf: function(){ return "\x00"}}
   assert(+a === 0) }
  \end{lstlisting}
  \normalsize
  %% \caption{Warning captured as \hi{} priority.}
  %% \label{fig:hi-priority}
  \vspace{-1ex}  
\end{figure}

The object property \CodeIn{valueOf} stores a function that returns a
primitive value identifying the target object~\cite{valueof}. The
original version of this code returns an empty string whereas the
version of the code modified by the \radamsa{} fuzzer~\cite{radamsa}
returns a string representation of a null character (\CodeIn{NUL}).%\Comment{ in ascii}
The unary plus expression ``\CodeIn{+a}", used in the
assertion, is equivalent to the operation
\CodeIn{ToNumber(a.valueOf())} that converts a string to a number,
otherwise the operation returns \gls{nan}~\cite{unary-plus}.
This test fails in all engines but
\chakra{}. For all three engines the string cannot be parsed as an
hexadecimal. As such, they produce a NaN as result and the test fails
as expected. \chakra{}, instead, incorrectly converts the string
to zero, making the test to pass. As Table~\ref{tab:bugs} shows, the
\chakra{} team fixed the issue soon after we reported the problem.

\vspace{1ex}\noindent\textbf{Issue \#18, Table~\ref{tab:bugs}.}  The
code {\footnotesize\ttfamily{eval(function b(a)\{break;\});}} shows
the test input we used to reveal a bug in \veight{}. This code snippet
was obtained by fuzzing a Test262 test with \quickfuzz. In its
original version, a string (omitted for space), passed as argument to
the \CodeIn{eval} function, encoded the actual test. The fuzzer
replaced the string argument with a function whose body is a
\CodeIn{break} statement outside a valid block statement. Section
B.3.3.3 from the spec~\cite{spec-b333} documents how eval should
handle code containing function declarations.  According to the
spec~\cite{break-statement}, the virtual machine should throw an early
error--in this case, a SyntaxError--if the break statement is not
nested in an iterable or a switch statement. All engines, but
\veight{}, behave as expected in this case.

%% \Igor{
%%   Este foi caso de bug confirmado que foi manifestado no engenho \veight{}.
%%   Este test case oriundo da suite Test262 foi fuzzado pelo \quickfuzz{}
%%   que por ser um fuzzer mutacional, usa uma gramatica \js{} que auxilia na geracao
%%   de entradas validas. In Figure~\ref{fig:bug-veight}, we observed a \CodeIn{eval} statement
%%   that try to evaluate a \CodeIn{function f()}, this test case specify the ecma spec
%%   B.3.3.3~\cite{spec-b333} that describe an evaluation of a block statement in eval code containing a 
%%   function declaration. The fuzzer removes the function declaration and 
%%   injects a function with a break statement outside a loop. According to specification~\cite{break-statement},
%%   the break statement should throw an early error (in this case a SyntaxError) 
%%   if the break statement is not nested in an iterational or a switch statement. All engines works
%%   as expected but \veight{} pass without failures.
%% }


\vspace{1ex}\noindent\textbf{Issue \#18,
  Table~\ref{tab:test-transplantation-bugs}.}  The code snippet below
shows the test input we used to reveal a bug in \jsc{}. This test
originates from \jerry{} test suite; the bug was found during the test
transplantation experiment.

\begin{figure}[h!]
  \vspace{-0.5ex}  
  \centering
  \scriptsize
  \lstset{escapeinside={@}{@},
%    numbers=left,xleftmargin=1em,frame=single,framexleftmargin=0.5em,
    basicstyle=\ttfamily\scriptsize, boxpos=c,
    numberstyle=\tiny,
    morekeywords={assertEq, var, yield, in, function, 
    typeof, return, throw, new, Error, if},
  }
  \begin{lstlisting}
 var obj = {}; var arr = [];
 try { arr.sort(obj); assert(false);}
 catch (e) { assert(e instanceof TypeError); }
  \end{lstlisting}
  \normalsize
  %% \caption{Warning captured as \hi{} priority.}
  %% \label{fig:hi-priority}
  \vspace{-1ex}  
\end{figure}

According to the specification~\cite{ecmas262-array-sort}, the
parameter to the \CodeIn{Array.sort} function should be a comparable
object or an undefined value, otherwise it should throw a
\CodeIn{TypeError}. In this case, \jsc{} accepts a non-callable object
as argument to \CodeIn{sort} and the test fails in the subsequent
step. The other engines raise a \CodeIn{TypeError} as expected.

\vspace{1ex}\noindent\textbf{Issue \#2, Table~\ref{tab:bugs}.}  The
code snippet below shows an example input related to a bug report that
we issued to the \chakra\ development team, but they did not accept.

\begin{figure}[h!]
  \vspace{-0.5ex}  
  \centering
  \scriptsize
  \lstset{escapeinside={@}{@},
%    numbers=left,xleftmargin=1em,frame=single,framexleftmargin=0.5em,
    basicstyle=\ttfamily\scriptsize, boxpos=c,
    numberstyle=\tiny,
    morekeywords={assertEq, var, yield, in, function, 
    typeof, return, throw, new, Error, if},
  }
  \begin{lstlisting}
 function test() {
   return typeof String.prototype.repeat === "function"
     && "foo".repeat(657604378) === "foofoofoo"; }
  \end{lstlisting}
  \normalsize
  %% \caption{Warning captured as \hi{} priority.}
  %% \label{fig:hi-priority}
  \vspace{-1ex}  
\end{figure}


This is a testcase original from the \jsc{} suite that the
\radamsa\ fuzzer modified. The original test used the integer literal
3 as argument to \CodeIn{repeat()}, but this test uses a long integer
instead. As result, the engine crashes. The team answered that this
was an incompatibility by design as the function was not expected to
receive such a long value.