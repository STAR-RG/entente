
\chapter{Preliminaries}
\label{sec:review}


\lipsum[4]

\section{Classifier and Clustering}
\lipsum[4]
\subsection{Naive Bayes}
\lipsum[4]
\section{Change Dectetion and Value Estimation}
\lipsum[4]

\begin{figure}[ht!]
\centering

\caption{Centro de Informática}
\label{fig:cap2:cinco-estrelas}
\fcolorbox{gray}{white}{\includegraphics[width=0.67\textwidth]{images/book-cover}}

\par\medskip\textbf{Fonte:} \cite{bruno:2015} \par\medskip
\end{figure}
\FloatBarrier





\begin{table}[ht]
\centering
\label{tab:tresleias}
\par\medskip\textbf{Fonte:} adaptado de \citeonline{bruno:2015} \par\medskip
\resizebox{0.95\textwidth}{!}{\begin{minipage}{\textwidth}
\centering
    \begin{tabular}{cp{10cm}} %{0.8\textwidth}{c|c|}
        
        \cline{1-2}
        %\multicolumn{2}{|c|c|}{Sets & eee} \\
        %\cline{1-2}        
        \toprule
        
        \textbf{Lei} & \textbf{Detalhe}  \\
        \hline \hline        
        1 & 
        Se o dado não pode ser encontrado e indexado na \textit{Web}, ele não existe. \\ \hline                
        2 & 
        Se não estiver aberto e disponível em formato compreensível por máquina, ele não pode ser reaproveitado. \\ \hline     
        3 &
        Se algum dispositivo legal não permitir sua replicação, ele não é útil. \\ %\hline        
        \bottomrule        
        %\multicolumn{2}{|c|c|}{Sets & eee} \\
       % \hline
    \end{tabular}
\end{minipage} }
\caption{As Três Leis dos Dados Abertos Governamentais}
\end{table} 
\FloatBarrier


\lipsum[4]